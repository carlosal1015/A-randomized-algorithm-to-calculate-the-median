\documentclass[a4 paper,12pt]{article}
\usepackage[utf8]{inputenc}
\usepackage{enumerate}

\begin{document}
	\section*{ESTADO DEL ARTE}
	\begin{itemize}
		\item El primer algoritmo aleatorizado fue un método desarrollado por Michael O. Rabin para el problema de par más cercano en la geometría computacional:
		\begin{center}
			"Dados $n$ puntos en un espacio métrico, encuentre un par de puntos con la menor distancia entre ellos "
		\end{center}
		Un algoritmo ingénuo para encontrar distancias entre todos los pares de puntos en un espacio de dimensión $d$ y seleccionar el mínimo requiere $O(n^2)$ tiempo y se puede resolver en 
		$O(n\log n)$ tiempo en un espacio euclidiano. En el modelo computacional que asume que la función de piso es computable en tiempo constante, el problema se puede resolver en el tiempo $O(n\log(\log n) )$. Si permitimos que la aleatorización se use junto con la función de piso, el problema se puede resolver en el tiempo $O(n)$.
		\item Michael O. Rabin demostró que la prueba de primalidad de Miller de $1976$ se puede convertir en un algoritmo aleatorizado. En ese momento, no se conocía ningún algoritmo determinístico práctico para la primalidad.
		\item Actualmente existen varios algoritmos aleatorios para encontrar la mediana, cada algoritmo se diferencia en su procedimiento para ordenar los datos aleatorios. Los algoritmos más usados por su eficacia y eficiencia para ordenar datos son  mergesort,  heapsort y quicksort. 
	\end{itemize}
\end{document}