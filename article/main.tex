\documentclass[9pt,draft,a4paper,twoside,onecolumn,romanappendices]{IEEEtran}
\usepackage[utf8]{inputenc}
\usepackage[T1]{fontenc}
\usepackage[spanish]{babel}
\spanishdatedel
\usepackage{amsmath,amsthm}
\usepackage[cmintegrals]{newtxmath}
\usepackage{bm}
\usepackage{graphicx}

\theoremstyle{definition}
\newtheorem{definition}{Definición}[subsection]
\title{Un algoritmo aleatorizado para calcular la mediana}
\author{Jesús Jáuregui \and Miller Silva \and Erwin \and Carlos Aznarán\thanks{El profesor César Lara Ávila}}

\renewcommand\IEEEkeywordsname{Palabras clave}
\begin{document}
\maketitle
\begin{abstract}
En este trabajo, 
% Una pequeña descripción total del trabajo.
\end{abstract}
\begin{IEEEkeywords}
Cuantil, Percentil, Mediana, Quick Sort, Heap Sort.
\end{IEEEkeywords}
% TODO: Después de introducción que esté en dos columnas.
\section{Introducción}

\subsection{Definiciones}

% TODO: (JAUREGUI) Completar 
\begin{definition}[Percentil]
Es un valor
\end{definition}

\begin{definition}[Mediana de una distribución]
Es un valor $x_m$ de $\mathbb{X}$ tal que el $50\%$ de los posibles valores de $X$ están por debajo de $x_m$ y el $50\%$ de los posibles valores de $\mathbb{X}$ están por arriba de $x_m$.
\end{definition}

\section{Agradecimientos}

\end{document}